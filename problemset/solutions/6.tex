\documentclass{article}
\usepackage[utf8]{vietnam}
\usepackage{xcolor}

\usepackage{amsmath}
\usepackage{amssymb}

\title{Pima2021}
\author{Long Nguyen}
\date{July 26, 2021}
\begin{document}
\maketitle
\tableofcontents
\pagebreak
\section{Mô tả bài toán}
\subsection{Bài toán 1}
Dữ liệu được đưa dưới dạng một danh sách các vector $D$ chiều được ký hiệu là: $X = (\vec{x}_1, \vec{x}_2, \ldots, \vec{x}_n)^T$ với $\vec{x}_i \in \mathbb{R}^d$. \\ 

Một phân phối chuẩn nhiều chiều định nghĩa bởi vector trung bình và $\vec{\mu}$ ma trận covariance $\Sigma$. Vector ngẫu nhiên $\vec{X}$  được gọi là tuân theo phân phối đều $D$ chiều ký hiệu là: $\vec{X} \sim N_D(\vec{\mu}, \Sigma)$, khi đó hàm mật độ xác suất có thể được tính như công thức \ref{congthuc1}
\begin{equation}
    f(\vec{x}; \vec{\mu}, \Sigma)
    = \dfrac{1}{\sqrt{(2\pi)^k|\Sigma|}} \exp{-\dfrac{1}{2} (\vec{x} - \vec{\mu}) \Sigma^{-1} (\vec{x}-\vec{\mu})^T}
    \label{congthuc1}
\end{equation}
\subsection{Bài toán 2}
Thầy Dũng muốn tham dự trại hè Pima 2022 ở Cape Town, Nam Phi. Tuy nhiên, do không có đường bay thẳng từ Thành phố Hồ Chí Minh đến Nam Phi nên thầy Dũng phải quá cảnh ở hai thành phố khác. Dựa vào bảng sau đây, hãy giúp thầy Dũng chọn lộ trình bay ít tốn kém nhất.\\
\begin{table}[h]
    \flushleft
    \begin{tabular}{|l|c|c|c|c|c|c|}\hline
     {}&HCM&Chiangmai&Singapore&Santa Marta&San Antonio   \\ \hline
      {{\text{HCM}}}& - &{250}&{176}&{1039}& -    \\ \hline
      {{\text{Chiangmai}}}& - & - & - & - &{1480} \\ \hline
      {{\text{Singapore}}}& - & - & - & - &{1733}  \\ \hline
      {{\text{Santa Marta}}}& - & - & - & - &{540}  \\ \hline
      {{\text{San Antonio}}}& - & - & - & - & - \\ \hline
\end{tabular}
\end{table}
\end{document}